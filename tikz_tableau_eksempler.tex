%%% VIGTIGT. Skal kompileres med LuaLatex for at virke!

\documentclass[11pt,a4paper]{article}

\addtolength{\textheight}{2.5cm}
\addtolength{\topmargin}{-1.25cm}

\usepackage[danish]{babel}
\usepackage{amssymb}
\usepackage[utf8]{luainputenc}
\usepackage{url}

%%%% tikz er pakken som gør det hele muligt %%%%

\usepackage{tikz}
\usetikzlibrary{graphs,graphdrawing,quotes,babel} % vi har brug for nogle tikz-biblioteker
\usegdlibrary{trees, layered} % biblioteker til graf-placerings-algoritmer

%%%% et par kommandoer til at markere Όbne og lukkede grene %%%%

\newcommand{\closed}{child[white, level distance=5mm] {node[black] {$\times$}}}
\newcommand{\open}{child[white, level distance=5mm] {node[black] {$\bigcirc$}}}

%%%%% et par shortcuts for typesetting af de to sandhedsværdier %%%%%

\newcommand{\F}{\mathtt{F}}
\newcommand{\T}{\mathtt{T}}

%%%%%% navne på tableaureglerne: %%%%%%%%
%%  \tof er en forkortelse for "->F"
%% \lorf er en forkortelse for  "\/F"
%% \landf er en forkortelse for "/\F"
%% osv.


\newcommand{\lorf}{\ensuremath{\lor\F}}
\newcommand{\lort}{\ensuremath{\lor\T}}
\newcommand{\landf}{\ensuremath{\land\F}}
\newcommand{\landt}{\ensuremath{\land\T}}
\newcommand{\tof}{\ensuremath{\to\!\F}}
\newcommand{\tot}{\ensuremath{\to\!\T}}
\newcommand{\lrf}{\ensuremath{\leftrightarrow\!\F}}
\newcommand{\lrt}{\ensuremath{\leftrightarrow\!\T}}
\newcommand{\negf}{\ensuremath{\neg\F}}
\newcommand{\negt}{\ensuremath{\neg\T}}
\newcommand{\allf}{\ensuremath{\forall\F}}
\newcommand{\allt}{\ensuremath{\forall\T}}
\newcommand{\exf}{\ensuremath{\exists\F}}
\newcommand{\ext}{\ensuremath{\exists\T}}


%%%%%%% en kommando til at typesette formlerne i tableauer. Kommandoen kaldes med \formel{<nummer>}{<formel>}{<F hvis sættes falsk, T hvis sættes sand>} %%%%%%%%%%%%%%%%

\newcommand{\formel}[3]{{\color{red} \scriptsize #1} $#2: \ifx#3F \F \else \T \fi$}




\title{tikz tableau-eksempler i Latex}
\author{Thomas Bolander}
\date{\today}


\begin{document}
\maketitle

Vigtig information:
\begin{itemize}
  \item Dokumentet kan kun kompilere med LuaLatex. Det er fordi jeg gør brug af nogen pakker som kalder graf-placerings-algoritmer, som kun findes til LuaLatex. Fordelen ved dette er så til gengæld at man har et meget slagkraftigt værktøj, som selv kan placere knuderne i grafer og træer uden at man behøver specificere koordinater el.lign.
  \item Hvis du vil vide mere om de værktøjer jeg bruger, så kig på manualen til tikz og pgf: 
  \begin{center} \color{blue} \footnotesize
  \url{http://mirrors.dotsrc.org/ctan/graphics/pgf/base/doc/pgfmanual.pdf}
  \end{center}
  \item Tikz kan mange ting, som man kan se her:
  \begin{center} \color{blue} \footnotesize
  \url{http://www.texample.net/tikz/examples/}
  \end{center}
   Men det er nok ikke for alle at skulle bruge et tekstuelt format til at angive grafisk indhold. Det kræver noget tilvænning. Gå ikke i gang med at bruge tikz i latex til at lave tableauer, hvis du er under tidspres! Når man har vænnet sig til at bruge det kan det være ret effektivt, men i starten kan det godt være noget af en tidssluger.
  \item For at skære det ud i pap skal du kun kaste dig ud i at lave tableauer i Latex hvis: 1) du har god tid; 2) du synes det er sjovt at rode med den slags; 3) du forventer at skulle bruge tikz til andre ting i fremtiden (hvad mange af jer givetvis vil komme til).
  \item For at komme i gang kan du kigge på eksemplerne nedenfor samt de korte instruktioner jeg har skrevet som kommentarer i selve tex-dokumentet. 
\end{itemize}


\subsection*{Eksempel 1}

 \[
 \begin{tikzpicture}[align=center,every edge quotes/.style={magenta,auto,font=\scriptsize}]
   \graph[trie,tree layout,level sep=7mm,sibling distance=30mm] 
   {
   %% "--" bruges til at lave en kant. Så herunder har vi  først formlen (p->q)->q som vi sætter F, og dernæst en kant:
     "\formel{1}{((p \to q) \to q) \to q}{F}" -- 
     %% nu kommer der to formler lige under hinanden (begge konklusioner af samme regel), og dem adskilller vi med "\\". 
     %% Desuden skal vi markere navnet på reglen vi har brugt. Det gør vi med [> "\tof" på 1"]:
     "\formel{2}{(p \to q) \to q}{T} \\ \formel{3}{q}{F}" [> "$\tof$ på 1"] --  
     %% nu har vi brug for en forgrening. Derfor starter vi en Tuborg-parantes med to elementer, den venstre gren og den
     %% højre gren, separeret af et komma:
     {
       "\formel{4}{p \to q}{F}" --
       "\formel{6}{p}{T} \\ \formel{7}{q}{F} \\ \bigcirc" [>"$\tof$ på 4"] , 
       "\formel{5}{q}{T} \\ \times" [> "$\tot$ på 2",yshift=-2mm] 
     }
  };
  \end{tikzpicture}
\]

\subsection*{Eksempel 2}

\[
 \begin{tikzpicture}[align=center,every edge quotes/.style={magenta,auto,font=\scriptsize}]
   \graph[trie,tree layout,level sep=7mm,sibling distance=30mm] 
   {
     "\formel{1}{((q \to p) \to q) \to q}{F}" -- 
     "\formel{2}{(q \to p) \to q}{T} \\ \formel{3}{q}{F}" [> "$\tof$ på 1"] --  
     {
       "\formel{4}{q \to p}{F}" --
       "\formel{6}{q}{T} \\ \formel{7}{p}{F} \\ \times" [>"$\tof$ på 4"] , 
       "\formel{5}{q}{T} \\ \times" [> "$\tot$ på 2",yshift=-2mm] 
     }
  };
  \end{tikzpicture}
\]


\subsection*{Eksempel 3}

\[
  \begin{tikzpicture}[align=center,every edge quotes/.style={magenta,auto,font=\scriptsize}]
    \graph[trie,tree layout,level sep=7mm,sibling distance=30mm] 
    {  
      "\formel{1}{(p \to q) \wedge (p \to \neg q) \to \neg p}{F}" --
      "\formel{2}{(p \to q) \wedge (p \to \neg q)}{T} \\ \formel{3}{\neg p}{F}" [> "$\to\!\F$ på 1"] --
      "\formel{4}{p}{T}"[> "$\neg\F$ på 3"] --
      "\formel{5}{p \to q}{T} \\ \formel{6}{p \to \neg q}{T}" [> "$\land\top$ på 2"] --
      {
        "\formel{7}{p}{F} \\ \times"[yshift=-2mm]  ,
        "\formel{8}{q}{T}" [>  "$\to\T$ på 5" near end] -- 
        {
          "\formel{9}{p}{F} \\ \times"[yshift=-2mm] ,
          "\formel{10}{\neg q}{T}"  [> "$\to\T$ på 6" near end]  -- "\formel{11}{q}{F} \\ \times" [> "$\neg\top$ på 10"]
        }
      }
    };
  \end{tikzpicture}
\]

\subsection*{Eksempel 4}

\[
  \begin{tikzpicture}[align=center,every edge quotes/.style={magenta,auto,font=\scriptsize}]
    \graph[trie,tree layout,level sep=7mm,sibling distance=30mm] 
    {  
      "\formel{1}{ p \leftrightarrow ((q \leftrightarrow r) \leftrightarrow ((p \leftrightarrow q) \leftrightarrow (p \leftrightarrow r))) }{F}" --
      {
        "\formel{2}{\neg (p \wedge q)}{1} \\ \formel{3}{\neg p \vee \neg  q}{F}" --
        "\formel{6}{p \wedge q}{F}" [>"$\negt$ på 2"] -- 
        "\formel{7}{\neg p}{F} \\ \formel{8}{\neg q}{F}" [>"$\lorf$ på 3"] --
        "\formel{9}{p}{T}"  [> "$\negf$ på 7"] --
        "\formel{10}{q}{T}" [> "$\negf$ på 8"] --
        {
          "\formel{11}{p}{F} \\ \times" ,
          "\formel{12}{q}{F} \\ \times" [> "$\landf$ på 6"]
        }
       ,
       "\formel{4}{\neg (p \wedge q)}{F} \\ \formel{5}{\neg p \vee \neg q}{T}" [> "$\lrf$ på 1" near end] --
       "\formel{13}{p \wedge q}{T}" [>"$\negf$ på 4"] --
       "\formel{14}{p}{T} \\ \formel{15}{q}{T}" [>"$\landt$ på 13"] --
       {
         "\formel{16}{\neg p}{T}" -- "\formel{18}{p}{F} \\ \times" [>"$\negt$ på 16"] ,
         "\formel{17}{\neg q}{T}" [>"$\lort$ på 5"] -- "\formel{19}{q}{F}  \\ \times" [>"$\negt$ på 17"]
       }
     }
   };
 \end{tikzpicture}
\]


\end{document}

  